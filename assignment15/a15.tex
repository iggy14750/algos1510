\documentclass{article}
\author{Isaac B Goss\\ James Hahn\\ Jonathan Dyer}
\title{Assignment 15}

\usepackage{amsmath}
\usepackage{amsthm}
\usepackage{enumitem}
\usepackage[margin=0.8in]{geometry}
\usepackage{graphicx}

% ============ USED FOR OUR FORMAT ============
\newtheorem{thm}{Claim}
\providecommand{\prob}[1]{\section*{Problem #1}}
\providecommand{\soln}{\textbf{Solution: }}
\providecommand{\image}[1]{
    \begin{center}
        \includegraphics[width=0.5\textwidth]
            {#1}
    \end{center}
}
\providecommand{\tightlist}{
    \setlength{\itemsep}{0pt}\setlength{\parskip}{0pt}
}

% ============ USED FOR CODE LISTINGS ============
\usepackage{listings}
\usepackage[usenames,dvipsnames,svgnames]{xcolor}
\definecolor{javagreen}{rgb}{0.25,0.5,0.35}
\lstset{
    basicstyle   = \footnotesize,
    commentstyle = \color{javagreen},
    frame        = single,
    language     = C,
    stringstyle  = \color{orange},
    numbers      = left,
    showstringspaces=false,
    deletekeywords = {len, max, format, min},
    morekeywords = {yield, function, then, do, to},
    keywordstyle = \color{blue},
    escapeinside={(*}{*)}
}


\begin{document}
\maketitle

\begin{prob} {23}
    Give a polynomial time algorithm for the following problem. The input consists of a sequence
    $R$ = $R_0$, \dots , $R_n$ of non-negative integers, and an integer $k$. The number $R_i$ represents the number
    of users requesting some particular piece of information at time $i$ ( say from a www server. If the
    server broadcasts this information at some time $t$, the the requests of all the users who requested the
    information strictly before time $t$ are satisfied. The server can broadcast this information at most
    $k$ times. The goal is to pick the $k$ times to broadcast in order to minimize the total time (over all
    requests) that requests/users have to wait in order to have their requests satisfied.\\
    
    Let's use the binary tree that we utilize for in-class examples to show our thought process for this problem.
    
    \image{treeOfSets}
    
    At every level in this tree, we can either choose to add time $t$ to our set of times that we send out the piece of information (indicated by the right child of each node), or we can not add time $t$ to the set (indicated by the left child).
    
    To prune this tree, we have created two simple rules:
    \begin{enumerate}
        \item If, at any node, we have already broadcast $k$ times, then we may prune the right child of this node.
        \item If two sets at the same level have broadcast the same number of times, then we can choose the one with the least total wait time
        (Notice that total wait time is an equivalent problem to average wait time).
    \end{enumerate}

    \pagebreak
    \begin{lstlisting}
int A[i, j] = minium-total-response time of some stragey from time 1 to time i,
    having broadcast j times. Dimensions n+2 (*$\times$*) k+1.
int L[i, j] = last-broadcast-time corresponding to the same node as in A. Same dimentions.
A[0, 0] = L[0, 0] = 0

for i = 0 to n do
    for j = 0 to n do

    \end{lstlisting}
    
    As one can see, our program incorporates two for loops with $k$ iterations for every $n$ to produce a runtime of $O(nk)$.  In the worst case, $n = k$, so the runtime is effectively $O(n^2)$.  This is indeed a polynomial runtime and solves our problem.
\end{prob}


\end{document}

























