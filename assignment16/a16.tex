\documentclass{article}
\author{Isaac B Goss\\ James Hahn\\ Jonathan Dyer}
\title{Assignment 16}
\date{Tuesday, October 10,  2017}

\usepackage{amsmath}
\usepackage{amsthm}
\usepackage{enumitem}
\usepackage[margin=0.8in]{geometry}
\usepackage{graphicx}

% ============ USED FOR OUR FORMAT ============
\newtheorem{thm}{Claim}
\providecommand{\prob}[1]{\section*{Problem #1}}
\providecommand{\soln}{\textbf{Solution: }}
\providecommand{\image}[1]{
    \begin{center}
        \includegraphics%[width=0.95\textwidth]
            {#1}
    \end{center}
}
\providecommand{\tightlist}{
    \setlength{\itemsep}{0pt}\setlength{\parskip}{0pt}
}

% ============ USED FOR CODE LISTINGS ============
\usepackage{listings}
\usepackage[usenames,dvipsnames,svgnames]{xcolor}
\definecolor{javagreen}{rgb}{0.25,0.5,0.35}
\lstset{
    basicstyle   = \footnotesize,
    commentstyle = \color{javagreen},
    frame        = single,
    language     = C,
    stringstyle  = \color{orange},
    numbers      = left,
    showstringspaces=false,
    deletekeywords = {len, max, format, min},
    morekeywords = {yield, function, then, do, to},
    keywordstyle = \color{blue},
    escapeinside={(*}{*)}
}

\newcommand{\powerset}{\raisebox{.15\baselineskip}{\Large\ensuremath{\wp}}}


\begin{document}
\maketitle

    \prob{24}
    Consider the tree: at each level we are deciding whether to include some element.
    As this tree is $n$ levels deep, it has $2^n$ leaves.
    This tree represents the decision of what to include on some side of the scale, say the left side for discussion.
    We can infer the content of the right side to be $\{B_1, \dots, B_i\} - L$ at some level $i$ if $L$ is the subset on the left side of the scale.

    We can drastically reduce the size of this tree by observing the following rule:
    \begin{enumerate}
        \item If two feasible solutions are on the same level, 
        and if the weight of the left side of one is the same as either side of the other, 
        then one can prune one of them arbitrarily.
    \end{enumerate}
    We can include this rule because:
    \begin{enumerate}[label=\roman*)]
        \item Given the level, and weight of one side, we can calculate the weight of the other.
        \item If both solutions have the same pair of weights, and a global solution can be built off of one of them,
            then one can also be built off of the other. 
            [The rest of the solution is disjoint from these two partial solutions].
    \end{enumerate}
    Please note that this can be done with two uses of the scale.
    The other solution can be reconstructed from some stored information, 
    and the scale can be used to tell us if these two subsets are of equal weight.
    %Now we're going to consider the following code

    \begin{lstlisting}
subset A[n (*$\times$*) nL] // A[i, *] holds a list of subsets of the first i boxes which are feasible
    // solutions.  This list has a length of nL because at maximum, n elements can be on
    // one side of the scale, each with L weight. Although we do not know the weight of 
    // a collection of boxes directly, there will be at most nL distinct weights,
    // so this can store them all.
A[0, 0] = (*$\emptyset$*)
for i = 0 to n do
    for j = 0 to nL do
        if A[i, j] is defined then
            // include (*$B_{i+1}$*) on the left scale.
            t = A[i, j] (*$\cup$*) (*$\{B_{i+1}\}$*)
            for k = 0 to nL do
                if A[i+1, k] is undefined then
                    A[i+1, k] = t // This solution shape has not been seen.
                if A[i+1, k] == t then // Weigh against left side of existing solution.
                    break // There is already a solution of this shape.
                if (*$\{B_1,\dots,B_{i+1}\}$*) - A[i+1, k] == t then // Weigh against right side.
                    break
            // Do not include (*$B_{i+1}$*), put it on the right side.
            for k = 0 to nL do
                if A[i+1, k] is undefined then
                    A[i+1, k] = A[i, j]
                if A[i+1, k] == A[i, j] or (*$\{B_1,\dots,B_{i+1}\}$*) - A[i+1, k] == A[i, j] then
                    break
output True if A[n, 0] is defined else False
    \end{lstlisting}
As we can see, we're looking at something like $O(n(nL)^2) = O(n^3L^2)$

\pagebreak
\prob{27}
Imagine the tree for this problem to be putting the current chapter which we are considering into either 
the volume which we are currently considering, or closing this volume and putting this chapter into a new volume.
The base state is having chapter 1 in volume 1.
Consider the following pruning rules:
\begin{enumerate}
    \item We cannot move onto to more than $k$ volumes.
    \item If two feasible solutions are on the same level, 
    and are working on the same volume, 
    then choose the one with the better objective function \textbf{not including the current volume}.
\end{enumerate}
The reason the second can be chosen is that if some 
optimal solution exists building from the worse partial solution (measured as noted),
then we can build a similarly optimal solution from the better one.
We leave out the final volume because that one is in some sense ``in progress'', and more importantly, causes contradictions if it's included.
But of completed volumes, we have that any solution built off one cannot beat the existing one.
That is, in some partial solution, we have a greatest and least volume.
The objective function is defined by these volumes.
Any future solution built off of this one will have other volumes which may be:
\begin{enumerate}
    \item Shorter than the least from before, in which case we have a worse objective function.
    \item Between the least and greatest volumes from before, which does not affect the objective function, or
    \item Longer than the greatest from before, which again increases the objective function.
\end{enumerate}

Since the rest of such an optimal solution beats or matches the worst partial's objective function, 
we can safely say that building another solution from the better partial will also be optimal.

\begin{lstlisting}
list<int> A[n (*$\times$*) k] // A[i, j] = a sequence of volume lengths made from the first i chapters
    // of length j (having j volumes), minimizing the objective func of the first j-1 elements.
A[1, 1] = {(*$x_1$*)}

function obj(s):
    x = infinity
    n = -infinity
    for item in s[excluding last element] do
        x = max(x, item)
        n = min(n, item)
    return x - n

for i = 1 to n do
    for j = 1 to n do
        if A[i, j] is defined then
            // Don't open a new volume
            new = A[i, j] (*$\cup$*) {(*$x_{i+1}$*)}
            old = A[i+1, j]
            if obj(new) < obj(old) then
                A[i+1, j] = new
            // Open a new volume.
            old = A[i+1, j+1]
            if obj(new) < obj(old) then
                A[i+1, j+1] = new
output A[n, k]
\end{lstlisting}

\pagebreak
\prob{Reduction 1}
Let us propose the following implementation of matrix multiplication.
\begin{lstlisting}
function matrixMult(A, B):
    C = [[0, 0, 0],
         [0, 0, 0],
         [0, A, 0]]
    D = [[0, 0, 0],
         [B, 0, 0],
         [0, 0, 0]]
    E = lowerTriangeMult(C, D)
    return lower left(E)
\end{lstlisting}

\end{document}




















