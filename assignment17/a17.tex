\documentclass{article}
\author{Isaac B Goss\\ James Hahn\\ Jonathan Dyer}
\title{Assignment 17}
\date{October 11, 2017}

\usepackage{amsmath}
\usepackage{amsthm}
\usepackage{enumitem}
\usepackage[margin=0.8in]{geometry}
\usepackage{graphicx}

% ============ USED FOR OUR FORMAT ============
\newtheorem{thm}{Claim}
\providecommand{\prob}[1]{\section*{Problem #1}}
\providecommand{\soln}{\textbf{Solution: }}
\providecommand{\image}[1]{
    \begin{center}
        \includegraphics%[width=0.95\textwidth]
            {#1}
    \end{center}
}
\providecommand{\tightlist}{
    \setlength{\itemsep}{0pt}\setlength{\parskip}{0pt}
}

% ============ USED FOR CODE LISTINGS ============
\usepackage{listings}
\usepackage[usenames,dvipsnames,svgnames]{xcolor}
\definecolor{javagreen}{rgb}{0.25,0.5,0.35}
\lstset{
    basicstyle   = \footnotesize,
    commentstyle = \color{javagreen},
    frame        = single,
    language     = C,
    stringstyle  = \color{orange},
    numbers      = left,
    showstringspaces=false,
    deletekeywords = {len, max, format, min},
    morekeywords = {yield, function, then, do, to},
    keywordstyle = \color{blue},
    escapeinside={(*}{*)}
}


\begin{document}
\maketitle

\prob{2}
Here we wish to show that arbitrary matrix multiplication between matrices $A$ and $B$ is reducible to inverting an arbitrary nonsingular matrix $C$ (nonsingular of course meaning that such a matrix is in fact invertible). We can demonstrate this by way of the code given below, which you should consider in conjunction with the following:\par
\begin{itemize}
  \item The process of converting the two input matrices $A$ and $B$ to the input $C$ for 'matrixInverse' is polynomial, simply requiring the writing of a $3n*3n$ matrix with $9n^2$ entries, thus taking $O\left(n^2\right)$ time.
  \item The process of converting the output of 'matrixInverse' to the final output $AB$ is also polynomial, requiring only the retrieval of the top right $n*n$ entries of our $3n*3n$ result matrix; hence also $O\left(n^2\right)$ time.
  \item Since both of the above are $O\left(n^2\right)$ time operations, and our assumed 'matrixInverse' function takes at $least$ that much time, being in $O\left(n^k\right), k \geq 2$, then the overall algorithm for 'matrixMult' will be in $O\left(n^k\right)$ time, that being the greatest term in our runtime.
\end{itemize}


\begin{lstlisting}
function matrixMult(A, B):
    C = [[I, A, 0],
         [0, I, B],
         [0, 0, I]]
    /**
     * Note now that C' = C-inverse is
     * D = [[I, -A, AB],
     *      [0,  I, -B],
     *      [0,  0,  I]]
     * This will be the output of our function matrixInverse(C)
     */
    D = matrixInverse(C)
    return upper right(D)
\end{lstlisting}


\end{document}
