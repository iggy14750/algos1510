\documentclass{article}
\author{Isaac B Goss \& Jon Dyer}
\title{Assignment 2: Greedy Algorithms}
\date{Tuesday, August 30, 2017}

\usepackage{amsmath}
\usepackage{graphicx}
\usepackage{enumitem}
\usepackage{listings}
\usepackage[usenames,dvipsnames,svgnames]{xcolor}
\usepackage[margin=0.8in]{geometry}
\providecommand{\soln}{\textbf{Solution: }}
\providecommand{\image}[1]{
    \begin{center}
        \includegraphics{#1}
    \end{center}
}


\begin{document}
\maketitle

    \section*{Problem 1}
    Consider the following problem:
    
    INPUT: A set $S = \{(x_i, y_i)\ |\ 1 \leq i \leq n\}$ of intervals over the real line.
    
    OUTPUT: A maximum cardinality subset $S$ of $S$ such that no pair of intervals in $S$ overlap.
        
    Consider the following algorithm:
    
    \begin{lstlisting}
    Repeat until S is empty:
        1. Select the interval I that overlaps the least number of other intervals.
        2. Add I to final solution set S.
        3. Remove all intervals from S that overlap with I.
    \end{lstlisting}

    Prove or disprove that this algorithm solves the problem.\\
    
    \soln $\displaystyle \sum_{x \in Opt} \text{overlaps}(x) $
    
    \section*{Problem 2}
    
    Consider the following Interval Coloring Problem.\\
    INPUT:A set $S = \{(x_i, y_i)\ |\ 1 \leq i \leq n\}$ of intervals over the real line. Think of interval $(x_i, y_i)$ as being a request for a room for a class that meets from time $x_i$ to time $y_i$.\\
    OUTPUT: Find an assignment of classes to rooms that uses the fewest number of rooms.
    Note that every room request must be honored and that no two classes can use a room at the same
    time.
    
    \begin{enumerate}[label=\Alph*.]
        \item Consider the following iterative algorithm. Assign as many classes as possible to the first room (we can do this using the greedy algorithm discussed in class, and in the class notes), then assign as many classes as possible to the second room, then assign as many classes as possible to the third room, etc. Does this algorithm solve the Interval Coloring Problem? Justify your answer.
        
        \soln Consider the following set of intervals:
        
        \image{p2a_counterexample}
        
        Note that the optimal solution is to have $A,C$ in Room 1, and $B, D$ in Room 2.
        
        However, this algorithm outputs $A, D$ in Room 1, $B$ in Room 2, and $C$ in Room 3.
        
        First, it chooses $A$, eliminates $B$, chooses $D$, and then eliminates $C$. On the second pass, this algorithm chooses $B$, and eliminates $C$.
        Then on the third pass, only $C$ is left, and it is chosen.
        
        As this algorithm does not find the optimal solution, it does not solve this problem.
        
        \item Consider the following algorithm. Process the classes in increasing order of start times. Assume that you are processing class $C$. If there is a room $R$ such that $R$ has been assigned to an earlier class, and $C$ can be assigned to $R$ without overlapping previously assigned classes, then assign $C$ to $R$. Otherwise, put $C$ in a new room. Does this algorithm solve the Interval Coloring Problem?
        Justify your answer.
    \end{enumerate}
    
\end{document}

























