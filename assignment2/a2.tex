\documentclass{article}
\author{Isaac B Goss \& Jonathon Dyer}
\title{Assignment 2: Greedy Algorithms}
\date{Tuesday, August 30, 2017}

\usepackage{amsmath}
\usepackage{graphicx}
\usepackage{enumitem}
\usepackage{listings}
\usepackage[usenames,dvipsnames,svgnames]{xcolor}
\usepackage[margin=0.8in]{geometry}
\providecommand{\soln}{\textbf{Solution: }}
\providecommand{\image}[1]{
    \begin{center}
        \includegraphics{#1}
    \end{center}
}


\begin{document}
\maketitle

    \section*{Problem 1}
    Consider the following problem:
    
    INPUT: A set $S = \{(x_i, y_i)\ |\ 1 \leq i \leq n\}$ of intervals over the real line.
    
    OUTPUT: A maximum cardinality subset $S$ of $S$ such that no pair of intervals in $S$ overlap.
        
    Consider the following algorithm:
    
    \begin{lstlisting}
    Repeat until S is empty:
        1. Select the interval I that overlaps the least number of other intervals.
        2. Add I to final solution set S.
        3. Remove all intervals from S that overlap with I.
    \end{lstlisting}

    Prove or disprove that this algorithm solves the problem.\\
    
    \soln Let $A$ be the specified algorithm, and $A(I)$ be the output of running $A$ on input $I$.
    Assume, to get a contradiction, that $A$ does not solve this problem.
    That is, suppose for some input $I$, $A(I)$ in not optimal.
    Let $Opt$ be an optimal solution that is most like $A(I)$.
    They cannot be the same, or $A(I)$ would indeed be optimal.
    Consider any region of difference (RoD) where $A(I)$ differs from $Opt$.
    Within such a region, $Opt$ must contain more intervals than $A(I)$:
    
    \begin{enumerate}
        \item If there are fewer, then $A(I)$ has more disjoint intervals than $Opt$, contradicting optimality of $Opt$.
        \item If they are the same, then $A(I)$ has the same cardinality as $Opt$, contradicting the non-optimality of $A(I)$.
        \item If there are more in this region but fewer in another, then either 1 or 2 still applies but over the entire solution, leading to the same contradictions. Alternatively, as claimed, there are more intervals within the arbitrary RoD we've selected.
    \end{enumerate}
    
    Because there are more intervals in $Opt$ than $A(I)$, we have the following within the RoD:
    
    \begin{enumerate}
        \item There cannot exist any interval in $Opt$ that is disjoint from all in $A(I)$. This would contradict the algorithm, since $A$ would have chosen such an interval.
        \item There cannot exist any interval in $Opt$ that has fewer overlaps than every interval in $A(I)$. This would again contradict the algorithm, since $A$ would have chosen such an interval, it having fewer overlaps.
        \item So every interval in $Opt$ must have at least as many overlaps as the minimum overlapping interval in $A(I)$, noted $\min(A(I))$.
        That is, $ \forall \text{ intervals } x \in Opt $:
        $$ \text{ overlaps}(x) \geq \text{ overlaps}(min(A(I))) $$
    \end{enumerate}
    
    Because there are strictly more intervals in $Opt$ than $A(I)$ and each has \textbf{at least} the number of overlaps as min($A(I)$), it follows that, within the RoD, the total set of intervals in $Opt$ overlaps more than does the total set of set of intervals in $A(I)$.
    That is, $$ \sum_{x \in Opt} \text{overlaps}(x) > \sum_{y \in A(I)} \text{overlaps}(y) $$
    
    Because there are more overlaps in $Opt$ than in $A(I)$, there are necessarily fewer remaining intervals once those in $Opt$ have been selected.
    In this case, there must be 0 remaining (else $Opt$ would not be optimal).
    Then because the selection of intervals in $A(I)$ has overlapped (and hence eliminated) strictly fewer other intervals, there must be some nonzero number of intervals left for $A$ to select.
    This contradicts algorithm $A$ -- if we included those leftover intervals in $A(I)$, then $A(I)$ would ultimately match the cardinality of $Opt$, once again contradicting the non-optimality of $A(I)$.
    
    In any case, we have derived a contradiction, proving that $A$ does in fact solve the problem.
    
    \section*{Problem 2}
    
    Consider the following Interval Coloring Problem.\\
    INPUT:A set $S = \{(x_i, y_i)\ |\ 1 \leq i \leq n\}$ of intervals over the real line. Think of interval $(x_i, y_i)$ as being a request for a room for a class that meets from time $x_i$ to time $y_i$.\\
    OUTPUT: Find an assignment of classes to rooms that uses the fewest number of rooms.
    Note that every room request must be honored and that no two classes can use a room at the same
    time.
    
    \begin{enumerate}[label=\Alph*.]
        \item Consider the following iterative algorithm. Assign as many classes as possible to the first room (we can do this using the greedy algorithm discussed in class, and in the class notes), then assign as many classes as possible to the second room, then assign as many classes as possible to the third room, etc. Does this algorithm solve the Interval Coloring Problem? Justify your answer.
        
        \soln Consider the following set of intervals:
        
        \image{p2a_counterexample}
        
        Note that the optimal solution is to have $A,C$ in Room 1, and $B, D$ in Room 2.
        
        However, this algorithm outputs $A, D$ in Room 1, $B$ in Room 2, and $C$ in Room 3.
        
        First, it chooses $A$, eliminates $B$, chooses $D$, and then eliminates $C$. On the second pass, this algorithm chooses $B$, and eliminates $C$.
        Then on the third pass, only $C$ is left, and it is chosen.
        
        As this algorithm does not find the optimal solution, it does not solve this problem.
        
        \item Consider the following algorithm. Process the classes in increasing order of start times. Assume that you are processing class $C$. If there is a room $R$ such that $R$ has been assigned to an earlier class, and $C$ can be assigned to $R$ without overlapping previously assigned classes, then assign $C$ to $R$. Otherwise, put $C$ in a new room. Does this algorithm solve the Interval Coloring Problem?
        Justify your answer.
        
        
        \begin{lstlisting}
For all intervals I in S (by ascending start times):
    For all rooms R in use:
        If I can fit in R:
            Put I in R.
    If no room could be found:
        Create a new room R.
        Put I in R.
        \end{lstlisting}
    \end{enumerate}
    
\end{document}

























