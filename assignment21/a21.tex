\documentclass{article}
\author{Isaac B Goss\\ James Hahn\\ Jonathan Dyer}
\title{Assignment 21}
\date{25 Oct 2017}

\usepackage{amsmath}
\usepackage{amsthm}
\usepackage{enumitem}
\usepackage[margin=0.8in]{geometry}
\usepackage{graphicx}

% ============ USED FOR OUR FORMAT ============
\newtheorem{thm}{Claim}
\providecommand{\prob}[1]{\section*{Problem #1}}
\providecommand{\soln}{\textbf{Solution: }}
\providecommand{\image}[1]{
    \begin{center}
        \includegraphics%[width=0.95\textwidth]
            {#1}
    \end{center}
}
\providecommand{\tightlist}{
    \setlength{\itemsep}{0pt}\setlength{\parskip}{0pt}
}
\providecommand{\reducible}[2]{
  \textbf{#1} $\leq$ \textbf{#2}
}

% ============ USED FOR CODE LISTINGS ============
\usepackage{listings}
\usepackage[usenames,dvipsnames,svgnames]{xcolor}
\definecolor{javagreen}{rgb}{0.25,0.5,0.35}
\lstset{
    basicstyle   = \footnotesize,
    commentstyle = \color{javagreen},
    frame        = single,
    language     = C,
    stringstyle  = \color{orange},
    numbers      = left,
    showstringspaces=false,
    deletekeywords = {len, max, format, min},
    morekeywords = {yield, function, then, do, to},
    keywordstyle = \color{blue},
    escapeinside = {(*}{*)},
    mathescape
}


\begin{document}
\maketitle

\prob{13}
\begin{enumerate}[label=(\alph*)]
 \item This is NP-hard. We will show \reducible{Clique}{$\dfrac{3}{4}$IndependentSet}.
       \begin{lstlisting}
function Clique(Graph G, int k)
    //want $\dfrac{3}{4}n$ to be at least k, since all k-cliques contain (k-1)-cliques, etc.
    while k $> \dfrac{3}{4}n$
        add some leaf vertex anywhere in G
    return 34Clique(G, k)
 \end{lstlisting}

 \item This is NP-hard. We will show \reducible{$\dfrac{3}{4}$Clique}{$\dfrac{3}{4}$IndependentSet}. This is identical to the \reducible{Clique}{IndependentSet} that we did previously.
       \begin{lstlisting}
function 34Clique(Graph G, int k)
    return 34IndependentSet(G$^C$, k)
  \end{lstlisting}

 \item This is NP-hard. We will show \reducible{Clique}{CliqueAndInd}
       \begin{lstlisting}
function Clique(Graph G, int k)
    let H = new independent set of k vertices
    if CliqueAndInd(G+H,k)  // if the known independent set plus G passes the AND test, 
        return 1            // then G$\ni$(k-clique)
  \end{lstlisting}

 \item This is NP-hard. We will show  \reducible{IndependentSet}{CliqueOrInd}

 \item This is NP-hard. We will show \reducible{$\dfrac{3}{4}$Clique}{$\dfrac{3}{4}$CliqueAndInd}.

 \item This is too hard. I'm done.
\end{enumerate}

\pagebreak
\prob{14}
Each clause of a CNF Boolean formula can be viewed as a linear inequality in a system thereof,
in the sense that \textbf{all} inequalities must be satisfied, 
just as \textbf{all} clauses in a CNF forumla must be.
Let's represent every variable in the CNF by a variable of the same name in our system.
Then, each vaiable can either be added to or subtracted from the inequality based on whether it is inverted in the respective CNF clause.

We want these variables to only be assigned 1 or -1 (representing logic true and false, respectively), so we need to add an extra condition for each variable.
Some options are that for each variable $x$, we could choose one of the following:
\begin{itemize}
  \item $-1 \leq x \leq 1$, but this still allows for assignment of intermediate values.
  \item $n^2 = 1$, but this is not a linear equation.
  \item $(-1)^x$ could be used as the variable to satisfy, but this of course is not linear either.
  \item $|x| = 1$ is the closest to linear, so this is what we'll use.
\end{itemize}

Each inequality then is satisfied when any of the terms evaluate to positive 1.
To accomplish this, we can put negative the number of terms in the equation.

Let's look at an example to explain this.
Consider the following CNF forumla.

\begin{align*}
    (\overline{x} \lor y \lor z) \land (x \lor \overline{y} \lor z \lor w \lor \overline{u})
\end{align*}

This can be transformed into the following system of linear inequalities.

\begin{align*}
    -x + y + z &\geq 2\\
    x - y + z + w -u &\geq 4\\
    |x| = |y| = |z| = |w| = |u| &= 1
\end{align*}

Thus, this here is our ultimate reduction.

\begin{lstlisting}
function CNF-Decider(F)
    S = a system of linear inequalities
    for clause C in F do
        I = a new linear inequality, default "0 $\geq$ 0"
        for literal l in C do
            subtract 1 from right-hand-side of I
            let v = the variable in literal l
            if v has not been seen in S then
                add equality "|v| = 1" to S // make sure only 1 or -1 is assigned.
            if l = $\overline{v}$ then
                add term "-v" to left-hand-side of I
            else
                add term "+v" to left-hand-side of I
        add inequality I to S
    return LinearIneqDecider(S)
\end{lstlisting}

\end{document}
