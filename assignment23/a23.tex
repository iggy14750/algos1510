\documentclass{article}
\author{Isaac B Goss\\ James Hahn\\ Jonathan Dyer}
\title{Assignment 23}

\usepackage{amsmath}
\usepackage{amsthm}
\usepackage{enumitem}
\usepackage[margin=0.8in]{geometry}
\usepackage{graphicx}

% ============ USED FOR OUR FORMAT ============
\newtheorem{thm}{Claim}
\providecommand{\prob}[1]{\section*{Problem #1}}
\providecommand{\soln}{\textbf{Solution: }}
\providecommand{\image}[1]{
    \begin{center}
        \includegraphics%[width=0.95\textwidth]
            {#1}
    \end{center}
}
\providecommand{\tightlist}{
    \setlength{\itemsep}{0pt}\setlength{\parskip}{0pt}
}
\providecommand{\reducible}[2]{
  \textbf{#1} $\leq$ \textbf{#2}
}

% ============ USED FOR CODE LISTINGS ============
\usepackage{listings}
\usepackage[usenames,dvipsnames,svgnames]{xcolor}
\definecolor{javagreen}{rgb}{0.25,0.5,0.35}
\lstset{
    basicstyle   = \footnotesize,
    commentstyle = \color{javagreen},
    frame        = single,
    language     = C,
    stringstyle  = \color{orange},
    numbers      = left,
    showstringspaces=false,
    deletekeywords = {len, max, format, min},
    morekeywords = {yield, function, then, do, to},
    keywordstyle = \color{blue},
    escapeinside={(*}{*)},
    mathescape
}


\begin{document}
\maketitle

\section*{Reduction}
\prob{21}
We tried. Truly we did.

\bigskip

\section*{Parallel}
\prob{1}
\begin{itemize}
  \item When $p = n$, this is identical to the problem done in class. We have:
  \begin{lstlisting}
AND($x_1 \dots x_n$, p)
    if p==1 then return $x_1$
    else return AND($x_1 \dots x_{\frac{n}{2}}$, p) && AND($x_{\frac{n}{2}+1} \dots x_n$)
  \end{lstlisting}
  \begin{itemize}
    \item As in class, we get $T(n,p) = T(\frac{n}{2},\frac{p}{2}) + 1$ which means this algorithm runs in $O(lg(n))$ time.
    \item The efficiency is $E(n,p) = \frac{n}{n*lg(n)}$ = $\frac{1}{lg(n)}$
    \item With $n^{\frac{1}{3}}$ processors, we get that $T(n,n^{\frac{1}{3}}) \leq n^{\frac{2}{3}}*lg(n)$ = $O(n^{\frac{2}{3}}*lg(n))$
  \end{itemize}

  \item When $p = \frac{n}{lg(n)}$ we have almost the same code as above:
  \begin{lstlisting}
logAND($x_1 \dots x_n$, p)
    if p==1 then return SequentialAND($x_1 \dots x_n$)
    else return logAND($x_1 \dots x_{\frac{n}{2}}$, p) && logAND($x_{\frac{n}{2}+1} \dots x_n$)
  \end{lstlisting}
  \begin{itemize}
    \item We showed in class that each leaf in the call tree for these kinds of algorithm will run in $\frac{n}{p}$ time, which in this case is $\frac{n}{\frac{n}{lg(n)}} = lg(n)$ time. So the overall $\frac{n}{p} + lg(n)$ that we usually have becomes $lg(n) + lg(n) \rightarrow O(lg(n))$ time.
    \item The efficiency is $\frac{n}{\frac{n}{lg(n)}*lg(n)} = 1$
    \item The upper bound for $n^{\frac{1}{3}}$ processors is $T(n,n^{\frac{1}{3}}) \leq \frac{n^{\frac{2}{3}}}{lg(n)}*T(n,\frac{n}{lg(n)}) = n^{\frac{2}{3}} \rightarrow O(n^{\frac{2}{3}})$
  \end{itemize}

  \item When we have a CRCW machine with $n$ processors we simply give every processor its own bit and have them all check if it's a zero. Using the 'common' write model, we have each of those processors write a 0 to the same location if and only if that processor's bit is zero (as checked above). Otherwise don't write anything.
  \begin{itemize}
    \item The above algorithm obviously runs in $O(1)$ time, since it just takes one check and write for each processor.
    \item The efficiency is $E(n,p) = \frac{n}{n*1} = 1$.
    \item With $n^{\frac{2}{3}}$ processors we get an upper bound of $n^{\frac{1}{3}}*T(n,n) \rightarrow O(n^{\frac{1}{3}})$
  \end{itemize}
\end{itemize}




\end{document}
