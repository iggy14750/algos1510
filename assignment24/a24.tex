\documentclass{article}
\author{Isaac B Goss\\ James Hahn\\ Jonathan Dyer}
\title{Assignment 24}

\usepackage{amsmath}
\usepackage{amsthm}
\usepackage{enumitem}
\usepackage[margin=0.8in]{geometry}
\usepackage{graphicx}

% ============ USED FOR OUR FORMAT ============
\newtheorem{thm}{Claim}
\providecommand{\prob}[1]{\section*{Problem #1}}
\providecommand{\soln}{\textbf{Solution: }}
\providecommand{\image}[1]{
    \begin{center}
        \includegraphics%[width=0.95\textwidth]
            {#1}
    \end{center}
}
\providecommand{\tightlist}{
    \setlength{\itemsep}{0pt}\setlength{\parskip}{0pt}
}
\providecommand{\reducible}[2]{
  \textbf{#1} $\leq$ \textbf{#2}
}

% ============ USED FOR CODE LISTINGS ============
\usepackage{listings}
\usepackage[usenames,dvipsnames,svgnames]{xcolor}
\definecolor{javagreen}{rgb}{0.25,0.5,0.35}
\lstset{
    basicstyle   = \footnotesize,
    commentstyle = \color{javagreen},
    frame        = single,
    language     = C,
    stringstyle  = \color{orange},
    numbers      = left,
    showstringspaces=false,
    deletekeywords = {len, max, format, min},
    morekeywords = {yield, function, then, do, to},
    keywordstyle = \color{blue},
    escapeinside={(*}{*)},
    mathescape
}


\begin{document}
\maketitle

\section*{Reductions}
\prob{22}
We consider the following transformation from input of \textbf{Triangle Problem} to input of \textbf{Dr. Seuss}.
\begin{lstlisting}
function transform(U)
    A = T = $\emptyset$
    for triple t in U do
        $T_t$ = function(a) {
            if a == $x_t$ or a == $y_t$ or a == $z_t$ then
                return true
            else return false
        }
        add  $x_t$, $y_t$, and $z_t$ to A
    return A, T
\end{lstlisting}

Then the following is our reduction, notice that we invert the output:
\begin{lstlisting}
function Triangle(U)
    A, T = transform(U)
    return !DrSuess(A, T)
\end{lstlisting}

\begin{thm}
    This algorithm correctly reduces \textbf{Triangle} to \textbf{Dr. Seuss}
\end{thm}
\begin{proof} We need to satisfy incorrect input to  \textbf{Dr. Seuss} $\iff$ correct input to \textbf{Triangle}.

    \begin{enumerate}
        \item Incorrect input to \textbf{Dr. Seuss} $\implies$ correct input to \textbf{Triangle}
        Such an incorrect input would look like two diseases $a, b$ for which $T_i(a) = T_i(b)$ for all functions $T_i \in T$.
        That is, we have two variables of $X,Y,Z$ which are included in the same triple of $U$.
        This is perfectly valid input to \textbf{Triangle}.

        \item Correct input to \textbf{Triangle} $\implies$ incorrect input to \textbf{Dr. Seuss}.
        Then we have a collection of triples, each with distinct variables.
        In particular, there are 3 variables in the same triple which are in no other triple.
        This means that they have the exact same output for every function.
        This is incorrect input to \textbf{Dr. Suess}.
    \end{enumerate}
\end{proof}


\pagebreak
\section*{Parallel Algos}
\prob{2}


\prob{3}
The problem is to initialize an array of length $n$ with each cell filled with the integer $x$. The parts are each simple enough, and almost exactly the same as in our last assigment.
\begin{itemize}
  \item On an EREW PRAM with $n$ processors, the following algorithm runs in $\frac{n}{n} + lg(n) = O(lg(n))$ time:
  \begin{lstlisting}
ARR($A[1] \dots A[n], x, p$)
    if p==1 then $A[1] = x$
    else (*\(
          \left \{
              \begin{tabular}{l}
              ARR($A[1] \dots A[\frac{n}{2}], x, \frac{p}{2}$) \\
              ARR($A[\frac{n}{2}+1] \dots A[n], x, \frac{p}{2}$)
              \end{tabular}
          \right \}
          \)*)
  \end{lstlisting}
  \begin{itemize}
    \item The efficiency is $E(n,n) = \frac{n}{n*lg(n)} = \frac{1}{lg(n)}$
    \item With $n^{\frac{1}{3}}$ processors we get that $T(n,n^{\frac{1}{3}}) \leq n^{\frac{2}{3}}*lg(n) \rightarrow O(n^{\frac{2}{3}}*lg(n))$
  \end{itemize}

  \item On an EREW PRAM with $\frac{n}{lg(n)}$ processors, the following alg runs in $lg(n) + lg(n) \rightarrow O(lg(n))$ time:
  \begin{lstlisting}
logARR($A[1] \dots A[n], x, p$)
    if p==1 then SequentialARR($A[1] \dots A[n]$)
    else (*\(
          \left \{
              \begin{tabular}{l}
              logARR($A[1] \dots A[\frac{n}{2}], x, \frac{p}{2}$) \\
              logARR($A[\frac{n}{2}+1] \dots A[n], x, \frac{p}{2}$)
              \end{tabular}
          \right \}
          \)*)
  \end{lstlisting}
  \begin{itemize}
    \item The efficiency is $\frac{n}{\frac{n}{lg(n)}*lg(n)} = 1$
    \item The upper bound for $n^{\frac{1}{3}}$ processors is $T(n,n^{\frac{1}{3}}) \leq \frac{n^{\frac{2}{3}}}{lg(n)}*T(n,\frac{n}{lg(n)}) = n^{\frac{2}{3}} \rightarrow O(n^{\frac{2}{3}})$
  \end{itemize}

  \item This algorithm is trivial: concurrently read $x$ with each processor, then concurrently write $x$ into $A[i]$ for each $i \in [n]$
  \begin{itemize}
    \item The efficiency is $E(n,p) = E(n,n) = \frac{n}{n*1} = 1$.
    \item With $n^{\frac{2}{3}}$ processors we get an upper bound of $T(n, n^{\frac{2}{3}}) \leq n^{\frac{1}{3}}*T(n,n) \rightarrow O(n^{\frac{1}{3}})$
  \end{itemize}
\end{itemize}

\end{document}
