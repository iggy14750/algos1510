\documentclass{article}
\author{Isaac B Goss\\ James Hahn\\ Jonathan Dyer}
\title{Assignment 25}

\usepackage{amsmath}
\usepackage{amsthm}
\usepackage{enumitem}
\usepackage[margin=0.8in]{geometry}
\usepackage{graphicx}

% ============ USED FOR OUR FORMAT ============
\newtheorem{thm}{Claim}
\newtheorem{lemma}{Lemma}
\providecommand{\prob}[1]{\section*{Problem #1}}
\providecommand{\soln}{\textbf{Solution: }}
\providecommand{\image}[1]{
    \begin{center}
        \includegraphics%[width=0.95\textwidth]
            {#1}
    \end{center}
}
\providecommand{\tightlist}{
    \setlength{\itemsep}{0pt}\setlength{\parskip}{0pt}
}
\providecommand{\reducible}[2]{
  \textbf{#1} $\leq$ \textbf{#2}
}

% ============ USED FOR CODE LISTINGS ============
\usepackage{listings}
\usepackage[usenames,dvipsnames,svgnames]{xcolor}
\definecolor{javagreen}{rgb}{0.25,0.5,0.35}
\lstset{
    basicstyle   = \footnotesize,
    commentstyle = \color{javagreen},
    frame        = single,
    language     = C,
    stringstyle  = \color{orange},
    numbers      = left,
    showstringspaces=false,
    deletekeywords = {len, max, format, min},
    morekeywords = {yield, function, then, do, to},
    keywordstyle = \color{blue},
    escapeinside={(*}{*)},
    mathescape
}


\begin{document}
\maketitle

    \prob{23}
    We can say that \reducible{Vertex Cover}{River Crossing} by the following reduction:
    \begin{lstlisting}
function VertexCover(G, k)
    return RiverCrossing(G, k)
    \end{lstlisting}
    \begin{proof}
        We must establish that a $k$-cover is possible iff a $k$-crossing is possible.
        \begin{itemize}
            \item $k$-cover impossible $\implies$ $k$-crossing impossible.
            If this is the case, then we cannot make the first move.
            Attempting to take any subset of vertices without breaking up every edge means we have lost the game.
            
            \item $k$-cover possible $\implies$ $k$-crossing possible.
            \begin{lemma}
                If a graph $G$ has a $k$-cover, then all subgraphs of $G$ contain a $k$-cover.
            \end{lemma}
            
            This tells us that it is always possible to collect a subset of vertices so that either side of the river is edge-free.
            
        \end{itemize}
    \end{proof}
    
    \prob{5}
    On an EREW PRAM with $n$ processors, we get the following algorithm for computing $n!$ in $O(lg(n))$ time:\\
    \begin{lstlisting}
        fact($k, n, p$)   // k is the starting value for our partial product
        if p == 1 then return 1
        else return fact($k, \frac{n+k}{2}, \frac{p}{2}$) $\times$ fact($\frac{n+k}{2}, n, \frac{p}{2}$)
    \end{lstlisting}
    
\end{document}