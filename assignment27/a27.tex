\documentclass{article}
\author{Isaac B Goss\\ James Hahn\\ Jonathan Dyer}
\title{Assignment 27}

\usepackage{amsmath}
\usepackage{amsthm}
\usepackage{enumitem}
\usepackage[margin=0.8in]{geometry}
\usepackage{graphicx}

% ============ USED FOR OUR FORMAT ============
\newtheorem{thm}{Claim}
\providecommand{\prob}[1]{\section*{Problem #1}}
\providecommand{\soln}{\textbf{Solution: }}
\providecommand{\image}[1]{
    \begin{center}
        \includegraphics%[width=0.95\textwidth]
            {#1}
    \end{center}
}
\providecommand{\tightlist}{
    \setlength{\itemsep}{0pt}\setlength{\parskip}{0pt}
}
\providecommand{\reducible}[2]{
  \textbf{#1} $\leq$ \textbf{#2}
}
\usepackage{mathtools}
\DeclarePairedDelimiter\ceil{\lceil}{\rceil}
\DeclarePairedDelimiter\floor{\lfloor}{\rfloor}

% ============ USED FOR CODE LISTINGS ============
\usepackage{listings}
\usepackage[usenames,dvipsnames,svgnames]{xcolor}
\definecolor{javagreen}{rgb}{0.25,0.5,0.35}
\lstset{
    basicstyle   = \footnotesize,
    commentstyle = \color{javagreen},
    frame        = single,
    language     = C,
    stringstyle  = \color{orange},
    numbers      = left,
    showstringspaces=false,
    deletekeywords = {len, max, format, min},
    morekeywords = {yield, function, then, do, to},
    keywordstyle = \color{blue},
    escapeinside={(*}{*)},
    mathescape
}


\begin{document}
\maketitle

\prob{4}

\begin{lstlisting}
function ParPrefix($x_1, x_2, \dots, x_n$, p)
    if p == 1 then
        return SeqPrefix($x_1, x_2, \dots, x_n$)
    else
        A = new int[n]
        C = ParPrefix($x_2, x_4, \dots, x_n$, p/2) // Assume n to be even WLOG
        B = ParPrefix($x_1, x_3, \dots, x_{n-1}$, p/2)
        parFor i = 1 to n do
            A[i] = B[ceil(i/2)] + C[floor(i/2)]
        return A
\end{lstlisting}

Considering we have $p = \frac{n}{\log n}$ processors, this means that we have $T(n, p) = T(n/2, p/2) + \log n$, since that parallel for loop will have to schedule $\log n$ loop bodies to each processor.
We also have a base case of $T(n, 1) = \log n$.
Overall, we have $\Theta(\log^2 n)$.

\prob{7}
    \begin{lstlisting}
function EvaluatePoly($c_n, c_{n-1}, \dots, c_0$, p)
    if p == 1 then
        sequentially compute $c_nx^n + c_{n-1}x^{n-1} + \dots + c_1x^1 + c_0$
    else
        return EvaluatePoly($c_n, c_{n-1}, \dots, c_{\floor*{\frac{n}{2}}}$, p/2) + EvaluatePoly($c_{\floor*{\frac{n}{2}} - 1}, \dots, c_0$, p/2)
    \end{lstlisting}

    We assume that computing a single term $c_ix^i$ is a constant-time operation.
    This leaves us with $T(n, p) = T(n/2, p/2) + 1$ with the base case $T(n, 1) = \Theta(\frac{n}{n/\log n}) = \Theta(\log n)$ $\implies$ $\Theta(\log n)$ overall.
\end{document}
