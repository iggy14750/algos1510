\documentclass{article}
\author{Isaac B Goss\\ James Hahn\\ Jonathan Dyer}
\title{Assignment 27}

\usepackage{amsmath}
\usepackage{amsthm}
\usepackage{enumitem}
\usepackage[margin=0.8in]{geometry}
\usepackage{graphicx}

% ============ USED FOR OUR FORMAT ============
\newtheorem{thm}{Claim}
\providecommand{\prob}[1]{\section*{Problem #1}}
\providecommand{\soln}{\textbf{Solution: }}
\providecommand{\image}[1]{
    \begin{center}
        \includegraphics%[width=0.95\textwidth]
            {#1}
    \end{center}
}
\providecommand{\tightlist}{
    \setlength{\itemsep}{0pt}\setlength{\parskip}{0pt}
}
\providecommand{\reducible}[2]{
  \textbf{#1} $\leq$ \textbf{#2}
}
\usepackage{mathtools}
\DeclarePairedDelimiter\ceil{\lceil}{\rceil}
\DeclarePairedDelimiter\floor{\lfloor}{\rfloor}

% ============ USED FOR CODE LISTINGS ============
\usepackage{listings}
\usepackage[usenames,dvipsnames,svgnames]{xcolor}
\definecolor{javagreen}{rgb}{0.25,0.5,0.35}
\lstset{
    basicstyle   = \footnotesize,
    commentstyle = \color{javagreen},
    frame        = single,
    language     = C,
    stringstyle  = \color{orange},
    numbers      = left,
    showstringspaces=false,
    deletekeywords = {len, max, format, min},
    morekeywords = {yield, function, then, do, to},
    keywordstyle = \color{blue},
    escapeinside={(*}{*)},
    mathescape
}


\begin{document}
\maketitle

\prob{4}

\prob{7}
    \begin{lstlisting}
function EvaluatePoly($c_k, c_{k-1}, \dots, c_j$, p)
    if p == 1 then
        sequentially compute $c_kx^k + c_{k-1}x^{k-1} + \dots + c_jx^j$
    else
        return EvaluatePoly($c_k, c_{k-1}, \dots, c_{\floor*{\frac{k+j}{2}}}$, p/2) + 
            EvaluatePoly($c_{\floor*{\frac{k+j}{2}} - 1}, \dots c_j$, p/2)
    \end{lstlisting}

    We assume that computing a single term $c_kx^k$ is a constant-time operation.
    This leaves us with $T(n, p) = T(n/2, p/2) + 1$ with the base case $T(n, 1) = \Theta(\frac{n}{n/\log n}) = \Theta(\log n)$ $\implies$ $\Theta(\log n)$ overall.
\end{document}
