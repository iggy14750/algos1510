\documentclass{article}
\author{Isaac B Goss\\ James Hahn\\ Jonathan Dyer}
\date{November 10, 2017}
\title{Assignment 28}

\usepackage{amsmath}
\usepackage{amsthm}
\usepackage{enumitem}
\usepackage[margin=0.8in]{geometry}
\usepackage{graphicx}

% ============ USED FOR OUR FORMAT ============
\newtheorem{thm}{Claim}
\providecommand{\prob}[1]{\section*{Problem #1}}
\providecommand{\soln}{\textbf{Solution: }}
\providecommand{\image}[1]{
    \begin{center}
        \includegraphics%[width=0.95\textwidth]
        {#1}
    \end{center}
}
\providecommand{\tightlist}{
    \setlength{\itemsep}{0pt}\setlength{\parskip}{0pt}
}
\providecommand{\reducible}[2]{
    \textbf{#1} $\leq$ \textbf{#2}
}
\usepackage{mathtools}
\DeclarePairedDelimiter\ceil{\lceil}{\rceil}
\DeclarePairedDelimiter\floor{\lfloor}{\rfloor}

% ============ USED FOR CODE LISTINGS ============
\usepackage{listings}
\usepackage[usenames,dvipsnames,svgnames]{xcolor}
\definecolor{javagreen}{rgb}{0.25,0.5,0.35}
\lstset{
    basicstyle   = \footnotesize,
    commentstyle = \color{javagreen},
    frame        = single,
    language     = C,
    stringstyle  = \color{orange},
    numbers      = left,
    showstringspaces=false,
    deletekeywords = {len, max, format, min},
    morekeywords = {yield, function, then, do, to},
    keywordstyle = \color{blue},
    escapeinside={(*}{*)},
    mathescape
}


\begin{document}
\maketitle
    \prob{11}
    We will consider constructing an array T, which holds the tristate (or temporary, if you like) carries.
    That is, we can in constant time determine for each index $i$ whether summing those bits will:
    \begin{enumerate}\tightlist
        \item definitely carry out (generate)
        \item definitely not carry out (halt)
        \item will carry out if there is a carry into it (propagate)
    \end{enumerate}
    This can be computed in constant time in the following way (n processors in parallel).
    \begin{lstlisting}
if A[i] and B[i] == 1 then
    T[i] = 1 // generate
else if (A[i] or B[i]) == 0 then
    T[i] = 0 // halt
else
    T[i] = -1 // propagate
    \end{lstlisting}
    Note, also, that if we knew the actual carries, C, then addition would be constant time as well.
    \begin{lstlisting}
Sum[i] = Cin[i] xor A[i] xor B[i]
    \end{lstlisting}

    So the problem, then, is some transformation T $\to$ C.
    Sequentially, we can perform something like this.
    \begin{lstlisting}
for i = 1 to n do
    if T[i] == 1 then
        C[i] = 1 // Carry was definitely generated.
    else if T[i] == 0 then
        C[i] = 0 // Halts any carries
    else
        C[i] = C[i-1] // propagate grabs the previous one, already computed.
    \end{lstlisting}
    Note that this is an associative operator (try a few to convince yourself).
    So this should be parallelizable.
    However, each new output value is dependent on previous values.
    This looks familiar -- Parallel Prefix!

    It is known that Parallel Prefix can be solved in $O(\log n)$ on $n$ CREW processors, and in $O(\log^2 n)$ on $n$ EREW processors (\textbf{these are the conditions of a and b}).
    Thus, our overall algorithm is the following

\begin{lstlisting}
function Add(A, B)
    T = generate T as described from A, B
    C = ParPrefix(T)
    parFor i = 1 to n do
        S[i] = C[i] xor A[i] xor B[i]
    return S
\end{lstlisting}

    \pagebreak
    \prob{12}
        The innermost loop can be reduced from $O(n)$ to $O(\log n)$ by introducing a new array, T.
        This holds the weights of all alternative paths, and the minimum of such a list can be taken in parallel log time.
        \begin{lstlisting}
T = new int[n]
parFor k = 1 to n do
    T[k] = A[i, k] + A[k, j]
A[i, j] = min(A[i, j], ParMin(T, p = n))
        \end{lstlisting}

    \prob{13}
        In this case, we can introduce \textbf{another} new matrix, P, which holds the actual path from each pair of vertices.
        We shall also modify the usual parallel minimum function to create \texttt{ParMinI} which returns the \textbf{index} of the minimum element in an array, rather than the value itself.
        \begin{lstlisting}
T = new int[n]
parFor k = 1 to n do
    T[k] = A[i, k] + A[k, j]
l = ParMinI(T, p = n)
if T[l] < A[i, j] then
    P[i, j] = P[i, l] + P[l, j]
    A[i, j] = T[l]
        \end{lstlisting}

\end{document}
