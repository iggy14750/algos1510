\documentclass{article}
\author{Isaac B Goss\\ James Hahn\\ Jonathan Dyer}
\title{Assignment 29}

\usepackage{amsmath}
\usepackage{amsthm}
\usepackage{enumitem}
\usepackage[margin=0.8in]{geometry}
\usepackage{graphicx}

% ============ USED FOR OUR FORMAT ============
\newtheorem{thm}{Claim}
\providecommand{\prob}[1]{\section*{Problem #1}}
\providecommand{\soln}{\textbf{Solution: }}
\providecommand{\image}[1]{
    \begin{center}
        \includegraphics%[width=0.95\textwidth]
            {#1}
    \end{center}
}
\providecommand{\tightlist}{
    \setlength{\itemsep}{0pt}\setlength{\parskip}{0pt}
}
\providecommand{\reducible}[2]{
  \textbf{#1} $\leq$ \textbf{#2}
}

% ============ USED FOR CODE LISTINGS ============
\usepackage{listings}
\usepackage[usenames,dvipsnames,svgnames]{xcolor}
\definecolor{javagreen}{rgb}{0.25,0.5,0.35}
\lstset{
    basicstyle   = \footnotesize,
    commentstyle = \color{javagreen},
    frame        = single,
    language     = C,
    stringstyle  = \color{orange},
    numbers      = left,
    showstringspaces=false,
    deletekeywords = {len, max, format, min},
    morekeywords = {yield, function, then, do, to},
    keywordstyle = \color{blue},
    escapeinside={(*}{*)},
    mathescape
}


\begin{document}
\maketitle

\prob{15}

\prob{16}
Let us call the elements of the right array $r_1, r_2, \dots, r_{n/2}$, and those of the left array $l_1, l_2, \dots, l_{n/2}$.
We assign to each of these elements $\frac{n}{2}$ processors, giving us $n * \frac{n}{2} = O(n^2)$ pocessors.
Now say that we are dealing with finding the position of $l_k$, without loss of generality.
Then, for starters, we know that at least $k-1$ elements (the $l_i$'s) will be to the left of $l_k$ in the merged array.
The question is how many of the $r_i$'s will also be to the left of $l_k$.

As stated, we have $\frac{n}{2}$ processors to do this, so each of these compares some $r_i$ against $l_k$, to determine if $r_i \leq l_k$.
The last element which satisfies this condition will be the last element to the left of $l_k$, meaning that the position of $l_k$ in the merged array will be $k - 1 + i$.
The question is, how do we know which of these was the last in constant time?
If we find that $r_i > l_k$, then we send a message to the processor comparing $l_k$ to $r_{i-1}$.
Only one processor will both not send such a message and recieve one, and this is the last $r_i$ which will be to the left of $l_k$.
With this, we can calculate the position of $l_k$ in the merged array.

\begin{lstlisting}
function ParMerge(L, R)
    A = new int[n]
    let $p_{k, i}$ = the processor merging in $l_k$ by comparing it to $r_i$, WLOG
    parFor k = 1 to n do
        parFor i = 1 to n/2 do
            if k $\leq$ n/2 then // the elements of L
                if L[k] < R[i] then
                    send a message to $p_{k, i-1}$
                else if $p_{k, i}$ recived a message then
                    A[k - 1 + i] = L[k]
            else // elements of R
                if R[k] < L[i] then
                    send a message to $p_{k, i-1}$
                else if $p_{k, i}$ recived a message then
                    A[k - 1 + i] = R[k]
    return A
\end{lstlisting}

\prob{17}

\end{document}
