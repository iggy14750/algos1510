\documentclass{article}
\author{Isaac B Goss\\ James Hahn\\ Jonathan Dyer}
\title{Assignment 29}

\usepackage{amsmath}
\usepackage{amsthm}
\usepackage{enumitem}
\usepackage[margin=0.8in]{geometry}
\usepackage{graphicx}

% ============ USED FOR OUR FORMAT ============
\newtheorem{thm}{Claim}
\newtheorem{lemma}{Lemma}
\providecommand{\prob}[1]{\section*{Problem #1}}
\providecommand{\soln}{\textbf{Solution: }}
\providecommand{\image}[1]{
    \begin{center}
        \includegraphics%[width=0.95\textwidth]
            {#1}
    \end{center}
}
\providecommand{\tightlist}{
    \setlength{\itemsep}{0pt}\setlength{\parskip}{0pt}
}
\providecommand{\reducible}[2]{
  \textbf{#1} $\leq$ \textbf{#2}
}

% ============ USED FOR CODE LISTINGS ============
\usepackage{listings}
\usepackage[usenames,dvipsnames,svgnames]{xcolor}
\definecolor{javagreen}{rgb}{0.25,0.5,0.35}
\lstset{
    basicstyle   = \footnotesize,
    commentstyle = \color{javagreen},
    frame        = single,
    language     = C,
    stringstyle  = \color{orange},
    numbers      = left,
    showstringspaces=false,
    deletekeywords = {len, max, format, min},
    morekeywords = {yield, function, then, do, to},
    keywordstyle = \color{blue},
    escapeinside={(*}{*)},
    mathescape
}


\begin{document}
\maketitle

\prob{15}
This is the minimum edit distance that we covered from the dynamic programming portion of the class to which we created an $n^3$ solution.  We need to create a poly-log time algorithm with a polynomial number of processors for this problem.

First, split the strings up in $O(logn)$ time; visually, this creates $n$ leaves in a tree for each string where each leaf is a letter of the string.  Compare each corresponding leaf node of each string.  If they match, great, we don't have to make any modifications because our minimum edit distance is cost 0 (we don't modify either substring).  Otherwise, create three copies of the "edit" string in $3*logn = O(logn)$ time, each representing one operation to the string (modify character, delete character, add character) and do this for every index of the substring (perform three operations for a one character substring, six operations for two-character, nine operations for three-character).  Now, in each tree, merge every sibling node in $O(logn)$ time; this is similar to our merge() function that we created in class for MergeSort().  Continue performing this exact same pattern and merge up the tree until we have reached the root with several cost values.  Choose the minimum cost value in $O(logn)$ time (this runtime should be pretty obvious with $n$ processors and $n$ items).

%This fits the CREW PRAM specification because we can concurrently read any character of the string we want.  However, it's exclusive write because we write our modified substrings to new locations in memory (via the copies as mentioned above).  This requires us to perform an $O(logn)$ copy, but allows us to have the exclusive write capability.

Finally, this has a polynomial number of processors.  As mentioned above, we only need $m$ processors for all $n$ substrings at the leaves to create copies, compare strings, and merge.  Therefore, for every $n$ substring, we require $m$ (approaches $n$ as we reach substrings of size $n$) processors.  As a result, we need $n*n = n^2$ processors.
\begin{lstlisting}
// Start = start string, Goal = string we want to reach
function MinEdit(Start s, Goal g, p):
    if p = n:
        if Start = Goal:
	    return (0, merge(Start, Goal)) // 0 cost, they already match
	else:
	    // Modify this character so they match
	    // Make a copy of the string and THEN modify it
            modCharString = modifyChar((*$s_1$*))
	
	    // Delete a character so they match
	    // Make a copy of the string and THEN delete a character
            delCharString = delChar((*$s_1$*))
			
	    // Add a character so the two strings match
	    // Make a copy of the string and THEN add a character
            addCharString = addChar(s)
            listOfSubstrings = [modCharString, delCharString, addCharString]
            return listOfSubstrings
    leftHalfSubstrings = MinEdit(Start (*$s_1 ... s_{n/2}$*), Goal (*$g_1 ... g_{k/2}$*), p/2)
    rightHalfSubstrings = MinEdit(Start (*$s_{n/2 + 1} ... s_n$*), Goal (*$g_{k/2 + 1} ... g_k$*), p/2)
	
    newSubstrings = list of substrings resulting from merging siblings in the tree
    // For parallel prefix, we can merge all the items in two lists in O(logn) time
    parFor each leftString, rightString in leftHalfSubstrings, rightHalfSubstrings:
        // stores a pair of the sum of the costs of both strings and the 
        // resultant merged string
        costStringPair = merge(leftString, rightString)
        newSubstrings.add(costStringPair)
    return newSubstrings
\end{lstlisting}
\pagebreak
\prob{16}
Let us call the elements of the right array $r_1, r_2, \dots, r_{n/2}$, and those of the left array $l_1, l_2, \dots, l_{n/2}$.
We assign to each of these elements $\frac{n}{2}$ processors, giving us $n * \frac{n}{2} = O(n^2)$ pocessors.
Now say that we are dealing with finding the position of $l_k$, without loss of generality.
Then, for starters, we know that at least $k-1$ elements (the $l_i$'s) will be to the left of $l_k$ in the merged array.
The question is how many of the $r_i$'s will also be to the left of $l_k$.

As stated, we have $\frac{n}{2}$ processors to do this, so each of these compares some $r_i$ against $l_k$, to determine if $r_i \leq l_k$.
The last element which satisfies this condition will be the last element to the left of $l_k$, meaning that the position of $l_k$ in the merged array will be $k - 1 + i$.
The question is, how do we know which of these was the last in constant time?
If we find that $r_i > l_k$, then we send a message to the processor comparing $l_k$ to $r_{i-1}$.
Only one processor will both not send such a message and recieve one, and this is the last $r_i$ which will be to the left of $l_k$.
With this, we can calculate the position of $l_k$ in the merged array.

\begin{lstlisting}
function ParMerge(L, R)
    A = new int[n]
    let $p_{k, i}$ = the processor merging in $l_k$ by comparing it to $r_i$, WLOG
    parFor k = 1 to n do
        parFor i = 1 to n/2 do
            if k $\leq$ n/2 then // the elements of L
                if L[k] < R[i] then
                    send a message to $p_{k, i-1}$
                else if $p_{k, i}$ recived a message then
                    A[k - 1 + i] = L[k]
            else // elements of R
                if R[k] < L[i] then
                    send a message to $p_{k, i-1}$
                else if $p_{k, i}$ recived a message then
                    A[k - 1 + i] = R[k]
    return A
\end{lstlisting}


\prob{17}
\begin{lemma}
    For any $x$, $\underbrace{\sqrt{\sqrt{\cdots\sqrt{x}}}}_{n\text{ times}}=x^{1/2^n}\leq \sqrt{2}$ iff $n = \lceil\log_2(\log_2(x))+1\rceil$
\end{lemma}
\begin{proof}
    See this: https://math.stackexchange.com/questions/129446/square-root-of-a-square-root-of-a-square-root
\end{proof}

That is, we can reduce $n$ elements into a constant number of elements in $O(\log \log n)$ steps.

\begin{lstlisting}
function ParMax2(X[1...n], p = n)
    if len(X) == 1 then
        return X[1]
    Sub = new int[$\sqrt{n}$]
    parFor i = 1 to $\sqrt{n}$ do
        k = i * $\sqrt{n}$ // plus or minus something...
        Sub[i] = ParMax2(X[k ... k + $\sqrt{n}$], $\sqrt{p}$)
    return ParMax1(Sub, p) // The algorithm from class. Note that $\sqrt{p}$ = |Sub|
\end{lstlisting}

Note that ParMax1 is simply the 'minimum' algorithm (for $k$ elements using $k^2$ procs, equivalently $\sqrt{k}$ elements with $k$ procs) that we discussed in class (performed in constant time) except simply replace all 'minimum' with 'maximum'. Thus the recursive call tree for the algorithm above has internal nodes that happen in constant time. And of course leaf operations also all happen at constant time (base case is size 1). Also, as noted above this subdivision of the problem (via square roots) happens in $O(\log \log n)$ steps. Thus our tree overall has $\log \log n$ levels, each of which happens in (parallel) constant time, giving us the desired runtime. 


\end{document}
