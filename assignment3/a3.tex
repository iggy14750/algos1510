\documentclass{article}
\author{Isaac B Goss \& Jonathan Dyer}
\title{Assignment 3: Greedy Algorithms}
\date{Friday, September 1, 2017}

\usepackage{amsmath}
\usepackage{graphicx}
\usepackage{enumitem}
\usepackage{listings}
\usepackage{amsthm}
\usepackage[usenames,dvipsnames,svgnames]{xcolor}
\usepackage[margin=0.8in]{geometry}
\newtheorem{thm}{Claim}
\providecommand{\soln}{\textbf{Solution: }}
\providecommand{\image}[1]{
    \begin{center}
        \includegraphics{#1}
    \end{center}
}
\providecommand{\prob}[1]{\section*{Problem #1}}

\begin{document}
\maketitle

    \prob{1}
    You wish to drive from point $A$ to point $B$ along a highway minimizing the time that you are stopped for gas. You are told beforehand the capacity $C$ of your gas tank in liters, you rate $F$ of fuel consumption in liters/kilometer, the rate $r$ in liters/minute at which you can fill your tank at a gas station, and locations $A = x_1, \dots, B = x_n$ of the gas stations along the highway. So if you stop to fill your tank from 2 liters to 8 liters, you would have to stop for $6/r$ minutes. Consider the following two algorithms. For each, prove or disprove that each correctly solve the problem.
    
    \begin{enumerate}[label=(\alph*)]
        \item Stop at every gas station, and fill the tank with just enough gas to make it to the next gas station.
        
        \begin{thm}
            This algorithm, Fill-as-you-go ($FAYG$), is optimal.
        \end{thm}
        \begin{proof}
            Assume, to get a contradiction, that $FAYG$ is not optimal.
            That is, assume that for some input $I$, $FAYG$ produces a non-optimal output.
            Let $FAYG(I)$ be such an output, and let $Opt(I)$ be an optimal output which agrees with $FAYG(I)$ for a maximum number of steps.
            
            Consider the first point of disagreement between $Opt(I)$ and $FAYG(I)$.
            This would be some stop $x_i$ where the two algorithms take different amounts of time filling up.
            Since $FAYG$ always gets just enough fuel to get to the next station, and $Opt$ and $FAYG$ have agreed up to this point, burning and obtaining the same amount of fuel, and $FAYG(x_i) \ne Opt(x_i)$, we have that $FAYG(x_i) < Opt(x_i)$
            
            Call the difference in time here $d$.
            Let's consider another output called $Opt'$ which is identical to $Opt$ in every way, except the following changes:
            
            \begin{enumerate}[label=\arabic*.]
                \item Let the time $Opt'(x_i) = Opt(x_i) - d = FAYG(x_i)$.
                \item Add the time $d$ to the following stop, so that $Opt'(x_{i+1}) = Opt(x_{i+1}) + d$ (unless this would overfill the tank, in which case space the remaining time out to the next stop, etc).
            \end{enumerate}
            
            We can easily see now that $Opt'(I)$ is still optimal.
            \begin{enumerate}[label=\arabic*.]
                \item There is still sufficient fuel to get from stop $x_i$ to $x_{i+1}$, since the amount $Opt'$ gets at that point matches exactly how much $FAYG(I)$ gets which is sufficient.
                \item The total amount of fuel consumed does not change, as the time $d$ is simply shifted to later visits.
            \end{enumerate}
            
            The only difference between the two is the $Opt'(I)$ agrees with $FAYG(I)$ for one more step than does $Opt(I)$.
            This is a contradiction on the maximal agreement between $Opt$ and $FAYG$.
            For this reason, $FAYG$ is optimal.
            
        \end{proof}
        
        
        \item Stop if and only if you don't have enough gas to make it to the next gas station, and if you stop, fill the tank up all the way.
    \end{enumerate}
    

    \prob{6}
    We consider a greedy algorithm for two related problems.
    
    \begin{lstlisting}
For i = 1 to n
    Place the ith word on the current line if it fits
        else place the ith word on a new line.
    \end{lstlisting}
    
    \begin{enumerate}[label=(\alph*)]
        \item The input to this problem consists of an ordered list of $n$ words. 
        The length of the $i$th word is $w_i$, that is the $i$th word takes up $w_i$ spaces. 
        (For simplicity assume that there are no spaces between words.)
        The goal is to break this ordered list of words into lines, this is called a layout. 
        Note that you can not reorder the words. 
        The length of a line is the sum of the lengths of the words on that line. 
        The ideal line length is $L$. 
        No line may be longer than $L$, although it may be shorter. 
        The penalty for having a line of length $K$ is $L − K$. 
        The total penalty is the sum of the line penalties.
        The problem is to find a layout that minimizes the total penalty.
        
        
        \item The input to this problem consists of a list of $n$ ordered words.
        The length of the $i$th word is $w_i$, that is the $i$th word takes up $w_i$ spaces.
        (For simplicity assume that there are no spaces between words.)
        The goal is to break this ordered list of words into lines, this is called a layout.
        Note that you cannot re-order the words.
        The length of a line is the sum of the lengths of the words on that line.
        The ideal line length is $L$.
        No line may be longer than $L$, although it may be shorter.
        The penalty for having a line of length $K$ is $L - K$.
        \emph{The total penalty is the \textbf{maximum} of the line penalties.}
        The problem is to find a layout that minimizes the total penalty.
        
        
    \end{enumerate}

\end{document}

























