\documentclass{article}
\author{Isaac B Goss\\ James Hahn\\ Jonathan Dyer}
\title{Assignment 30}

\usepackage{amsmath}
\usepackage{amsthm}
\usepackage{enumitem}
\usepackage[margin=0.8in]{geometry}
\usepackage{graphicx}

% ============ USED FOR OUR FORMAT ============
\newtheorem{thm}{Claim}
\providecommand{\prob}[1]{\section*{Problem #1}}
\providecommand{\soln}{\textbf{Solution: }}
\providecommand{\image}[1]{
    \begin{center}
        \includegraphics%[width=0.95\textwidth]
            {#1}
    \end{center}
}
\providecommand{\tightlist}{
    \setlength{\itemsep}{0pt}\setlength{\parskip}{0pt}
}
\providecommand{\reducible}[2]{
  \textbf{#1} $\leq$ \textbf{#2}
}

% ============ USED FOR CODE LISTINGS ============
\usepackage{listings}
\usepackage[usenames,dvipsnames,svgnames]{xcolor}
\definecolor{javagreen}{rgb}{0.25,0.5,0.35}
\lstset{
    basicstyle   = \footnotesize,
    commentstyle = \color{javagreen},
    frame        = single,
    language     = C,
    stringstyle  = \color{orange},
    numbers      = left,
    showstringspaces=false,
    deletekeywords = {len, max, format, min},
    morekeywords = {yield, function, then, do, to},
    keywordstyle = \color{blue},
    escapeinside={(*}{*)},
    mathescape
}


\begin{document}
\maketitle
\prob{18}
We're given a list of integers in the range 1 to $n$, and a CRCW Priority PRAM with $n$ processors. The following algorithm clearly runs in constant time (assumes that $p_1$ is the proc with highest priority, $p_n$ with lowest):
\begin{lstlisting}
function MAX($x_1, \dots, x_n$):
    T = new int[n]
    int max
    parFor i = 1 to n do
        // each processor puts a 1 in T, indexed by $x$-value
        arbitrarily assign procs $p_1 \dots p_n$ to the inputs, one each
        T[$x_i$] = 1
    parFor i = n to 1 do
        // now assign procs to cells in T in reverse order
        $p_i$ is assigned to T[n-i+1]
        if T[i] == 1 then
            max = T[i]    // highest priority proc with a '1' will be at index of max
    return max            // so that's the value that will get written
\end{lstlisting}

\prob{20}

The major issue in this case is advoiding concurrent reads.
This will arise when two positions of B are equal, and we'll neively read from the same position at the same time.
What we're going to do to prevent that is to ``collect'' all such positions of B with the problem (or, processors trying to write the same value).
We will then implement a log-time copy operation, such that the value is first written to two of the positions in C.
Then, each of these (which each have processors assigned) will write to 2 other positions in C, and so on.
It is possible to include some more optimization on that, but this gives us our polynomial time.

Let us say that processor $p_i$ is assigned position $i$ in A and C.
The question then is how we might collect all of these processors.
\begin{lstlisting}
function Compose(A, B, C, P, lo, hi)
    if lo == hi then
        begin copy operation described above.
    else
        mid = (lo + hi) / 2
        Compose(A, B, C, {$p_i \in P$ | B[i] $\leq$ mid}, lo, mid)
        Compose(A, B, C, {$p_i \in P$ | B[i] > mid}, mid, hi)
\end{lstlisting}

Then, the collect portion is log-time, leaving us with $O(\log n)$.

\end{document}
