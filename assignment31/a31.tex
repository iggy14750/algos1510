\documentclass{article}
\author{Isaac B Goss\\ James Hahn\\ Jonathan Dyer}
\title{Assignment 31}

\usepackage{amsmath}
\usepackage{amsthm}
\usepackage{amsfonts}
\usepackage{enumitem}
\usepackage[margin=0.8in]{geometry}
\usepackage{graphicx}

% ============ USED FOR OUR FORMAT ============
\newtheorem{thm}{Claim}
\providecommand{\prob}[1]{\section*{Problem #1}}
\providecommand{\soln}{\textbf{Solution: }}
\providecommand{\image}[1]{
    \begin{center}
        \includegraphics%[width=0.95\textwidth]
            {#1}
    \end{center}
}
\providecommand{\tightlist}{
    \setlength{\itemsep}{0pt}\setlength{\parskip}{0pt}
}
\providecommand{\reducible}[2]{
  \textbf{#1} $\leq$ \textbf{#2}
}
\providecommand{\R}{\mathbb{R}}

% ============ USED FOR CODE LISTINGS ============
\usepackage{listings}
\usepackage[usenames,dvipsnames,svgnames]{xcolor}
\definecolor{javagreen}{rgb}{0.25,0.5,0.35}
\lstset{
    basicstyle   = \footnotesize,
    commentstyle = \color{javagreen},
    frame        = single,
    language     = C,
    stringstyle  = \color{orange},
    numbers      = left,
    showstringspaces=false,
    deletekeywords = {len, max, format, min},
    morekeywords = {yield, function, then, do, to},
    keywordstyle = \color{blue},
    escapeinside={(*}{*)},
    mathescape
}


\begin{document}
\maketitle

\prob{23}
This is quite similar to the problem described in class, of reducing an expression tree of addition and subtraction operations.
The major difference lies in the addition of multiplication and division as operatations.
This seems to have no effect on the algorithm discussed in class, except when performing contractions on the tree.

When working with addition and subtraction, we knew that every edge function was of the form $\pm x + c$ ($c \in \R$), and that combining two of these functions with either of our operations would leave the result of the same form.
That is, we knew that our functions were closed under the relevant operations.

The problem, then, is with multiplication.
When multiplying two functions of this form, we will have a quadratic function.
And that, in general, the size of these functions can then, in the worst case, grow to a degree linear in $n$.
So that is clearly unacceptable.

However, this problem will only arise on multiplication operations of two non-leaf children.
But, as discussed in class, reducing in such a situation doesn't make sense anyway.
For multiplication, this decision avoids the production of arbitrarily-growing polynomials.

In the case that one or both children are leaves, the actual value can be calculated in constant time.
That is, suppose we chose to reduce a node with two children.
One was a non-leaf, with the function $f(x)$ on the edge to it.
The other was a leaf with a value $k$, and a function $g(x)$ on the edge to it.
Then $g(k)$ can be calculated in constant time, and $g(k)*f(x)$ will still be a linear polynomial
(what we see here is that now our functions are general linear formulae, but this does not affect our algorithm or its runtime).

So, in this way, we can apply the exact same algorithm from class to success here.

\end{document}
