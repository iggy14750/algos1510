\documentclass{article}
\author{Isaac B Goss\\ James Hahn\\ Jonathan Dyer}
\title{Assignment 4: Greedy Algorithms}
\date{Monday, September 11, 2017}

\usepackage{amsmath}
\usepackage{amsthm}
\usepackage{enumitem}
\usepackage[margin=0.8in]{geometry}
\usepackage{graphicx}
\usepackage{listings}
\usepackage[usenames,dvipsnames,svgnames]{xcolor}

\newtheorem{thm}{Claim}
\providecommand{\prob}[1]{\section*{Problem #1}}
\providecommand{\image}[1]{
    \begin{center}
        \includegraphics{#1}
    \end{center}
}


\begin{document}
\maketitle

    \prob{9}
    The input consists of $n$ skiers with heights $p_1,\dots,p_n$ and $n$ skies with heights $s_1,\dots,s_n$.
    The problem is to assign each skier to a ski to minimize the average difference between the height of a skier and his/her assigned ski.
    That is, if the $i$th skier is given the $\alpha(i)$th ski, then we want to minimize
    $$ \frac{1}{n} \sum_{i=1}^n |p_i - s_{\alpha(i)} | $$
    
    \begin{enumerate}[label=(\alph*)]
        \item Consider the following greedy algorithm.
        \begin{lstlisting}
        Repeat this process until every skier has a ski:
            Find the skier and ski whose height difference is minimized.
            Assign the skier this ski.
        \end{lstlisting}
        Prove or disprove this algorithm correct.
        
        \begin{thm}
            This algorithm is incorrect.
        \end{thm}
        \begin{proof}
            Consider the following counter-example.
            \image{p9table}
        \end{proof}
        
        \item Consider the following greedy algorithm.
        \begin{lstlisting}
        Sort the people and skies in order of increasing height.
        Assign to each person the ski in the same position in the ordering.
        \end{lstlisting}
        Prove or disprove this algorithm correct.
        
        \begin{thm}
            This algorithm, A, is indeed correct.
        \end{thm}
        \begin{proof}
            Assume, to reach a contradiction, that it does not.
            Then, there would be some input I on which A does not produce an optimal output.
            Consider some optimal solution Opt(I) which agrees with A(I) for the greatest number of steps of any other optimal solution.
            
            Let's call the first point of difference the $i$th person and ski (by A's ordering).
            In all solutions, we keep the people stable, and note the change by moving the skies.
            Ski $s_i$ is thus replaced in Opt(I) by some other ski $s_j$.
            Note that $i < j$, because $s_j$ has not yet appeared in A(I), $i$ being the first point of disagreement.
            This implies also that $s_i < s_j$ (Note that when comparing skies or people, we are considering their height).
            This is a strict inequality because $s_i$ was replaced by a \emph{different} ski.
            
            Now, $s_i$ must be somewhere else in Opt(I).
            As $i$ is the first point of disagreement, we can say that $s_i$ must have been assigned to some later person $p_k$ ($j$ is not necessarily equal to $k$).
            By saying that $i < k$, we have also said that $p_i \leq p_k$.

            Now, we construct a new solution, Opt'(I), which agrees with Opt(I) for all cases except for two.
            There is a ``swap'' between $s_i$ and $s_j$ in Opt'(I) from Opt(I), so that $s_i$ is re-assigned to $p_i$, and $s_j$ is assigned to $p_k$.
            It is clear that Opt'(I) agrees with A(I) for one more step than Opt(I).
            Consult the following picture of the situation.
            
            \begin{center}
            \begin{tabular}{r | p{0.22\textwidth}}
                A(I) & \ldots $p_{i-1}, p_i$ \ldots \newline
                    \ldots $s_{i-1}, s_i$ \ldots\\
                \hline
                Opt(I) & \ldots $p_{i-1}, p_i$ \ldots $p_k$ \ldots \newline
                \ldots $s_{i-1}, s_j$ \ldots $s_i$ \ldots \\
                \hline
                Opt'(I) & \ldots $p_{i-1}, p_i$ \ldots $p_k$ \ldots \newline
                \ldots $s_{i-1}, s_i$ \ldots $s_j$ \ldots \\
            \end{tabular}
            \end{center}
            
            By assumption, we have that Opt(I) is an optimal solution which agrees with A(I) for the maximal number of steps.
            Because Opt'(I) agrees with A(I) for one more step, it follows that it is not optimal.
            
            If this were the case, we would have that:
            $$ |p_i - s_j| + |p_k - s_i| < |p_i - s_i| + |p_k - s_j| $$
            
            We shall consider 6 cases to show that this cannot hold.
            \begin{enumerate}[label=Case \arabic*.]
                \item $ s_i < s_j \leq p_i \leq p_k $.
                \begin{align*}
                    (p_i - s_j) + (p_k - s_i) &< (p_i - s_i) + (p_k - s_j)\\
                    p_i + p_k - s_i - s_j &< p_i + p_k - s_i - s_j\\
                    0 &< 0
                \end{align*}
                
                \item $ s_i \leq p_i < s_j \leq p_k $.
                \begin{align*}
                    (s_j - p_i) + (p_k - s_i) &< (p_i - s_i) + (p_k - s_j)\\
                    s_j - p_i &< p_i - s_j\\
                    2s_j &< 2p_i\\
                    s_j &< p_i 
                \end{align*}
                
                \item $ p_i \leq s_i < s_j \leq p_k $.
                \begin{align*}
                    (s_j - p_i) + (p_k - s_i) &< (s_i - p_i) + (p_k - s_j)\\
                    s_j - s_i &< s_i - s_j\\
                    2s_j &< 2s_i\\
                    s_j &< s_i
                \end{align*}
                
                \item $ p_i \leq s_i \leq p_k < s_j $.
                \begin{align*}
                    (s_j - p_i) + (p_k - s_i) &< (s_i - p_i) + (s_j - p_k)\\
                    p_k - s_i &< s_i - p_k\\
                    2p_k &< 2s_i\\
                    p_k &< s_i
                \end{align*}
                
                \item $ p_i \leq p_k \leq s_i < s_j $.
                \begin{align*}
                    (s_j - p_i) + (s_i - p_k) &< (s_i - p_i) + (s_j - p_k)\\
                    0 &< 0
                \end{align*}
                
                \item $ s_i \leq p_i \leq p_k < s_j $.
                \begin{align*}
                    (s_j - p_i) + (p_k - s_i) &< (p_i - s_i) + (s_j - p_k)\\
                    p_k - p_i &< p_i - p_k\\
                    2p_k &< 2p_i\\
                    p_k &< p_i
                \end{align*}
            \end{enumerate}
            
            As we can see, it is impossible that this switch leaves Opt'(I) worse off than Opt(I).
            Thus, Opt'(I) is indeed optimal.
            This is a contradiction on the definition of Opt(I).
            Therefore, we have that A produces an optimal output on all input.
            That is, A is correct.
        \end{proof}
    \end{enumerate}
    
    \prob{10}
    INPUT: A collection of jobs $J_1, \dots, J_n$, where the $i$th job is a tuple $(r_i, x_i)$ of non-negative integers specifying the release time and size of the job.
    
    OUTPUT: A preemptive feasible schedule for these jobs on one processor that minimizes the total completion time $\sum_{i=1}^{n} C_i$.
    
    We consider two greedy algorithms for solving this problem that schedule in an online fashion, that is, algorithms of the following form.
    \begin{lstlisting}[escapeinside={(*}{*)}]
    t = 0
    While there are jobs left not completely scheduled:
        Among those jobs (*$J_i$*) such that (*$r_i \leq t$*), and that have previously 
        been scheduled for less than (*$x_i$*) time units:
            Pick a job (*$J_m$*) to schedule at time (*$t$*) according to some rule.
        Increment t.
    \end{lstlisting}
    
    One can get different greedy algorithms depending on the rule for selecting $J_m$. 
    For each of the following greedy algorithms, prove or disprove that the algorithm is correct. 
    Proofs of correctness must use an exchange argument.
    
    \textbf{SJF:} Pick $J_m$ to be the job with minimal size $x_i$.
    Ties are broken arbitrarily.
    \textbf{SRPT:} Let $y_{i,t}$ be the total time that job $J_m$ has been run before time $t$.
    Pick $J_m$ to be a job which has minimal remaining processing time, that is, that has minimal $x_i - y_{i, t}$.
    Ties broken arbitrarily.
    
    
    \begin{thm}
        
    \end{thm}

\end{document}

























