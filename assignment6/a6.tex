\documentclass{article}
\author{Isaac B Goss\\ James Hahn\\ Jonathan Dyer}
\title{Assignment 6}
\date{Friday, September 15, 2017}

\usepackage{amsmath}
\usepackage{amsthm}
\usepackage{enumitem}
\usepackage[margin=0.8in]{geometry}
\usepackage{graphicx}
\usepackage{listings}
\usepackage[usenames,dvipsnames,svgnames]{xcolor}

%\setlength{\parindent}{0pt}
\newtheorem{thm}{Claim}
\providecommand{\prob}[1]{\section*{Problem #1}}
\providecommand{\image}[1]{
    \begin{center}
        \includegraphics{#1}
    \end{center}
}


\begin{document}
\maketitle

	\prob{7}
	Consider the algorithm that selects the page that will NOT be accessed for the longest period of time in the future
	(this is the classic optimal page-replacement algorithm, requiring foreknowledge--which we here have), which we shall call "Not Used in the Future", or NUF. Suppose that it doesn't evict a page until the last possible moment (i.e. it is \textbf{lazy}).

	\begin{thm}
		This algorithm, NUF, correctly solves the problem of fast-slow memory page eviction (i.e. it minimizes the total \# of evictions).
	\end{thm}
	
	\begin{proof}
		Assume, to get a contradiction, that NUF is incorrect, that is, there exists some input $I$ for which NUF produces an output that is not optimal. Call this output N(I).
		Let Opt(I) be the output of an optimal algorithm which agrees with N(I) for the maximum number of steps.
		
		Recall that $k$ is the size (in pages) of fast memory and $n$ is the size (in pages) of slow memory.
		Note that we will presume that $k$ < $n$ because otherwise fast memory can simply hold a copy of every page and no evictions are necessary, obviating the problem. We also suppose that $k > 1$, since if $k = 1$ every request for a new page would result in an eviction, again obviating the problem. So there are at least 2 page slots in fast memory (and thus at least 3 in slow memory, so at least pages $a$, $b$, $c$ exist, else this problem is moot).
		
		Now, let $t$ be the point where Opt(I) and N(I) first disagree, in the sequence of memory access requests. At this point, let $a$ be the page evicted in N(I).  
		Let $b$ be the page evicted in Opt(I). We know $a \ne b$, else we are done.
		If $$t_i$$ is the length of time after $t$ where page $i$ is accessed again, then we know $t_a > t_b$ (else NUF would have chosen $b$).
		
		Now construct Opt'(I), which is identical to Opt(I), except that at time $t$ Opt'(I) evicts $a$ instead of $b$. Also the change according to cases below.
		It is clear that Opt'(I) is more similar to N(I) than Opt(I), so it remains to be shown that it is still optimal (i.e. that the \# of evictions has not increased).
		
		Because we specified that NUF is lazy, and since Opt(I) and Opt'(I) agree with N(I) up to time $t$, we know that there must be some page requested at time $t$ that is not in the $k$ pages already in fast memory, thus necessitating the eviction. Call this page $c$. Now consider our three outputs at time $t$: all have pages ($a$, $b$) in fast memory, and $c$ is being requested. Opt'(I) and N(I) evict $a$ to make space, so now the cache has ($b$, $c$), whereas Opt(I) evicts $b$ leaving ($a$, $c$). Note here that at this time $t+1$ or later, there must necessarily be another eviction/request of one of these ($a$ or $b$), else NUF and hence Opt' have the same number as Opt and we are done. We show that Opt'(I) is still optimal:
		
		\begin{enumerate}[label=\textbf{Case \arabic*.}]
			\item At time $u$, some eviction must occur and Opt(I) evicts $a$ in favor of $j$. Have Opt'(I) evict $b$, leaving both at ($j$, $c$).
			
			\item  At time $u$, a request to $b$ is made (this necessarily happens before a request to $a$ because of NUF). Then Opt(I) evicts $a$ to make room, giving ($b$, $c$). Then Opt'(I) can get from ($b$, $c$) (previous state) to that state in 1 or 0 evictions. Thus, the number of evictions in Opt'(I) $\leq$ the number of evictions in Opt(I) and it is equally optimal.
			
		 	\item At time $u$, a request to $b$ is made and Opt(I) chooses to evict $c$ (or whatever other page it has), leaving ($b$,$a$). Then we can do the same with Opt'(I) in 1 eviction.
		\end{enumerate}
	
		Generally, another way to put this scenario (after time $t$) is that Opt'(I) will only be non-optimal if it must evict somewhere Opt(I) doesn't. But the first time that would happen is either a request for $b$ (in which case (Opt'(I) does nothing and Opt(I) must evict, leaving Opt'(I) at least as well off) or a request for $a$ (in which case Opt'(I) would have to evict, but which necessarily comes after a request for $b$, leaving the two even in evictions). Other requests can use the other spot until such time as Opt'(I) is aligned with Opt(I) again.
	
		Thus in any case Opt'(I) is at least as optimal as Opt(I) \textbf{and} it agrees with N(I) for one more step. This is a contradiction, therefore N(I) is correct.
	\end{proof}

	\prob{17}
	Let $G = \{G_1, G_2, G_3, ..., G_n\}$ be the set of goods, and $G_H$, $G_W$ be the goods assigned to the husband and wife, respectively.
	
	Then we consider the algorithm:

	\begin{lstlisting}
Repeat until G is empty:
	W and H take turns choosing the good they have valued the highest,
		which is still part of G.
	They remove this good from G and place it in their own collections.
	\end{lstlisting}
	
	\begin{thm}
		This algorithm, A, produces a split which both W and H would find fair, if it exists.
	\end{thm}

	\begin{proof}
		Assume, to reach a contradiction, that this is not the case.
		Then there exists some input I on which some fair arrangement is possible, but A(I) is not fair.
		Let Opt(I) be a fair output which agrees with A(I) for the maximum number of steps taken by A of any other fair solution.
		
		Let $G_i$ be the first assignment on which A(I) and Opt(I) disagree.
		Say this assignment occurs at time t, and that $G_i$ was assigned to W (WLOG) in A(I), but to H in Opt(I) (not at time t).
		This means some other good, $G_k$, is assigned to W, where it was chosen by H in A(I).
		We know that $G_k$ is available at t because A(I) and Opt(I) agree up to t.
		Let $W(G_x)$ be the priority W assigned to $G_x$ (like n - it's position in W's ordering) (This applies to H as well).
		
		Now we construct Opt'(I), which is exactly like Opt(I), except that $G_i$ and $G_k$ have been swapped.
		Obviously, Opt'(I) agrees with A(I) for one more step than Opt(I).
		We know that W is happy, since $(G_i) > W(G_k)$ (otherwise, W would have chosen $G_k$ in A(I)).
		$G_k$ is assigned to H at some later time.
		
	\end{proof}


	\prob{1}
	Consider the recurrence relation $T(0) = T(1) = 2$ for $n > 1$
	
	\[ T(n) = \sum_{i=1}^{n-1} T(i)T(i-1) \]
	
	We consider the problem of computing $T(n)$ from $n$.
	
	\begin{enumerate}[label=(\alph*)]
		\item Show that if you implement this recursion directly in say the C programming language, that the
		program would use exponentially, in $n$, many arithmetic operations.

		
		\begin{proof}
			Implementing this recursively means that (only needing to consider the last iteration of the sum) $T(n)$ is made of two recursive-call subtrees of 2 different sizes, $T(n-1)$, and $T(n-2)$.
			$T(n-1)$ must be a larger subtree than $T(n-2)$, because namely $T(n-1)$ contains it.
			We know that this property is self-similar throughout all of the children.

			So it is that the complexity of $T(n)$ is bounded below by that of $2*T(n-2)$.
			If this pattern is continued until the base case of 1 or 0, then we would see $2*2*2*\dots*1 = 2^{n/2}$
			
			Thus, when only considering the final step of the sum, we find that the recursive-call tree, and therefore time complexity, of $T(n)$ is $O(2^n)$.
		\end{proof}
	
		\item Explain how, by not recomputing the same $T(i)$ value twice, one can obtain an algorithm for this
		problem that only uses $O(n^2)$ arithmetic operations.
		
		\begin{proof}
			Consider the following code, and its explanation.
			\begin{lstlisting}
Let T be an array of size n+1.
T[0] = T[1] = 2.
For i = 2, n do:
	int partial = 0.
	For j = 1, i-1 do:
		partial += T[j-1] * T[j].
	T[i] = partial.
return T[n].
			\end{lstlisting}
			Consider using a cache to store $T(i)$, $i\in [n]$.
			Then, to compute $T(n)$, we need only perform a linear amount of work, summing the partial products as defined above.
			We can use this to compute each partial value up to $n$.
			So, this linear work grows as we attack bigger problems, but notice that the rate it grows at is itself linear.
			The closed form of this is that overall, to compute $T(n)$ for some fixed $n$, requires $O(n^2)$ operations.
		\end{proof}
		
		\item Give an algorithm for this problem that only uses $O(n)$ arithmetic operations.
		
		\begin{proof}
			Consider the following code, and its explanation.
			\begin{lstlisting}
Let T be an array of size n+1.
T[0] = T[1] = 2.
Let sum S be the total sum so far.
For i = 1, n do:
	S += T[i-1] * T[i-2].
	T[i] = T[i-1] * T[i-2].
return T[n].
			\end{lstlisting}
			Consider using a cache to store $T(i)$, $i\in [n]$.
			Then, to compute $T(n)$, we need only perform a linear amount of work, summing the partial products as defined above.
			We can use this to compute each partial value up to $n$ requiring only one extra multiplication for every value $i$ that is computed in the summation.  Thus, if we passed in the value $n+1$, we have already computed $T(n)$ and $T(n-1)$ so the value of $T(n+1) = T(n)*T(n-1)$, one more simple multiplication.  Similarly, for $T(n+2)$, the value is one more multiplication of $T(n+1)*T(n)$.  This indicates that for every value $n$ that we increase, our number of arithmetic operations follows linearly and we end up with a runtime of $O(n)$.
		\end{proof}
	\end{enumerate}
	
	
\end{document}



































