\documentclass{article}
\author{Isaac B Goss\\ James Hahn\\ Jonathan Dyer}
\title{Assignment 8}
\date{Wednesday 20, 2017}

\usepackage{amsmath}
\usepackage{amsthm}
\usepackage{enumitem}
\usepackage[margin=0.8in]{geometry}
\usepackage{graphicx}

% ============ USED FOR OUR FORMAT ============
\newtheorem{thm}{Claim}
\providecommand{\prob}[1]{\section*{Problem #1}}
\providecommand{\soln}{\textbf{Solution: }}
\providecommand{\image}[1]{
    \begin{center}
        \includegraphics{#1}
    \end{center}
}
\providecommand{\tightlist}{
    \setlength{\itemsep}{0pt}\setlength{\parskip}{0pt}
}

% ============ USED FOR CODE LISTINGS ============
\usepackage{listings}
\usepackage[usenames,dvipsnames,svgnames]{xcolor}
\definecolor{javagreen}{rgb}{0.25,0.5,0.35}
\lstset{
    basicstyle   = \footnotesize,
    commentstyle = \color{javagreen},
    frame        = single,
    language     = C,
    stringstyle  = \color{orange},
    numbers      = left,
    showstringspaces=false,
    deletekeywords = {len, max, format, min},
    morekeywords = {yield, function, then, do, to},
    keywordstyle = \color{blue},
}


\begin{document}
\maketitle

    \prob{4}
    Give a dynamic programming algorithm to find the cheapest conversion from one string to another.
    
    \soln Given two strings A, B, begin at the final character of each.
    If these characters agree, then move on to the next one.
    If they don't, then try each of the following, and choose the option with lowest cost:
    \begin{enumerate}\tightlist
        \item Delete the current character from A, and continue on in A, keeping the same position in B.
        \item Insert the character from B into A, and continue on in B, keeping the same position in A.
        \item Replace the character from A with the one from B, and move onto the next character in each.
    \end{enumerate}

    Exhausting A will mean we have to insert all of the remaining characters from B, and reaching the end of B will mean that we have to delete all the remaining characters from A.
    
    Or, more succinctly:
    
    \begin{lstlisting}
function CSC(a, b):
    if a == 0 then return 4b // Insert the rest of B
    if b == 0 then return 3a // Delete the rest of A
    if A[a] == B[b] then
        return CSC(a-1, b-1)
    else
        return min(
            3 + CSC(a-1, b),  // Delete A[a]
            4 + CSC(a, b-1),  // Insert B[b]
            5 + CSC(a-1, b-1) // Replace A[a] <- B[b]
        )
    \end{lstlisting}

    Then we turn this into an iterative array-based algorithm.
    Since the input to this are 2 integers, we can use a 2-dimensional array.
    Consider using this table.
    A cell's value will depend in its neighbors above, to the left, and up-left.
    So, this table needs to be filled top-left to bottom-right.
    We recycle the same base cases from the recursive solution.
    Let n be the length of A, and m the length of B.
    
    \begin{lstlisting}
int CSC[n+1][m+1]
for b = 0 to m do
    CSC[0][b] = 3b
for a = 0 to n do
    CSC[a][0] = 4a

for a = 1 to n do
    for b = 1 to m do
        if A[a] == B[b] then
            CSC[a][b] = CSC[a-1][b-1]
        else
            CSC[a][b] = min(
                3 + CSC[a-1][b],
                4 + CSC[a][b-1],
                5 + CSC[a-1][b-1]
            )

return CSC[n][m]
    \end{lstlisting}
    
    \prob{5}
    Solve an instance of the minimum weight binary search tree problem using the algorithm described in class on the input keys (by weight) $(50, 5, 10, 20, 25)$.
    
    \image{jpic}

\end{document}

























